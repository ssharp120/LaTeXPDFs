\documentclass[10pt]{article}

\setlength{\parindent}{1.5eM}

\setlength{\parskip}{0ex}

\pagestyle{empty}

\title{MATH 351 Exercise 2}
\author{\textsc{\textit{Steven Sharp}}}
\date{}

\begin{document}

\maketitle

\thispagestyle{empty}

\begin{center}
\rule{\textwidth}{0.5pt}
\end{center}

During his 1948 correspondence with Lt. Frank O. Klein, R. Glenn Madill, Chief of Terrestrial Magnetism at the Department of Mines and Resources of Canada, concluded that \vspace{1ex}

``\ldots we agree on one point and that is the presence of what we can call the main magnetic pole on northwestern Prince of Wales Island. I have accepted as a purely preliminary value the position latitude \mbox{$73^{\circ}$ N} and longitude \mbox{$100^{\circ}$ W}. Your value of \mbox{$73^{\circ} 15'$ N} and \mbox{$99^{\circ} 45’$ W} is in excellent agreement, and I suggest that you use your value by all means.''

\begin{flushright}
--- \textsl{R. Glenn Madill, 21 July 1948} \vspace{1ex}
\end{flushright}

\setlength{\parindent}{1.5eM}Madill was indeed correct that the North \textit{magnetic} pole was located on Prince of Wales Island in 1948, but the Earth's magnetic poles are defined by locations where the local magnetic field has field lines that are aligned normal to the surface of the Earth at a given point; since the Earth's magnetic field is a dynamic result of the constant shifting of matter in the crust and mantle, the magnetic poles can cover large swaths of land area and have disjointed pockets of local North. Despite this, the majority of the magnetic North pole is concentrated in an area of a few hundred kilometers in radius at any given time. In the years since 1948, the magnetic North pole has continued to drift North in latitude and West in longitude, with models from 2020 placing the current location of the magnetic North pole at \mbox{$86^{\circ} 29' 38''$ N} and \mbox{$162^{\circ} 52' 1''$ E}. This is far closer to the geographic North pole than in the mid-20th century, but the path of the magnetic North pole indicates that it will likely start drifting further South in latitude in the coming decades.

While the displacement between the geographic and magnetic North pole may slightly degrade the accuracy of compass measurements for navigation, gyroscopic attitude determination devices and ubiquitous access to increasingly complex models of the Earth's magnetic field have removed the vast majority of instances of reliance on simple compass measurements in the modern age. In fact, sextants and astronomical navigation techniques have supplemented compass measurements for centuries, limited only by the working knowledge of cosmology and orbital mechanics at the time.

As humanity continues to develop cosmology and orbital mechanics, the dynamics of the Earth's magnetic field become more significant to the advancement of geological and space science. Of particular interest is modeling and simulating the Earth's magnetic field to develop methods of attitude control for Earth-orbiting spacecraft. Spacecraft may take advantage of the Earth's magnetic field both to determine their attitude and to impart torque on the vehicle at the expense of electrical power. However, detailed models of the Earth's magnetic field are also computationally expensive, as are the matrix operations required for coordinate translations and control laws. Therefore, for low-risk mission operations with lax pointing requirements, it would be advantageous to use a first-order approximation of the Earth's magnetic field as a simple magnetic dipole located at the center of mass of the Earth.

The \textit{geomagnetic} poles fulfill this exact purpose of determining the direction of the field lines from a theoretical bar magnet placed at the Earth's center of mass using a linear average of more detailed models of the Earth's magnetic field. The geomagnetic poles are purely theoretical and have no physical meaning outside of mathematical models; compass measurements will indicate the magnetic poles rather than the the geomagnetic ones. Regardless, it is still practical to determine an exact location of the geomagnetic poles, and 2020 models place the geomagnetic North pole at \mbox{$80^{\circ} 39' 0''$ N} and \mbox{$72^{\circ} 40' 48''$ W}. This means that the geomagnetic poles are located at an angular distance more than twice that of the magnetic poles away from the geographic North pole, although the angular rates are slightly reduced for the geomagnetic poles when compared to their counterparts.

\begin{center}
\rule{\textwidth}{0.5pt}
\end{center}

\end{document}
